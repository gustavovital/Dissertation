\chapter{\textbf{Introduction}}  \label{introduction}
\pagenumbering{arabic}

Economics decisions, in particular monetary policies, are based in resolutions provided by central banks. However, these resolutions are not always clear, especially for layman. Central banks minutes contain, for instance, much more information than it seems, particularly if a set of texts are considered for a conjunctural or structural analysis.\\

On occasion, the comprehension of a set of texts could be complex enough for a human to perform: the analysis may be referring to thousands or even millions of pages, depending on the purpose of the understanding. Nevertheless, problems like these are increasingly likely to be resolved due to advances in computational fields. Technics and tools such as Natural Language Processing (NLP) and text mining are being strongly used for a better comprehension and understanding of what a text, or a set of text (corpus), actually means or expresses.\\

Nowadays it is possible to recognize patterns on search pages as Google and Yahoo utilising text analysis in order to predict macroeconomics variables or even increase understanding of consumer behaviour. In recent research, Bholat, Hansen, Santos, \& Schonhardt-Bailey (2015) consider that even though the advances that have taken place in the field of computing are significant, the applicability of text mining in economics is still not used the way it could be.\\

An alternative approach of application still in the field of NLP is the sentiment analysis. Basically, it is “the task of identifying positive and negative opinions, emotions, and evaluations” (Wilson, Wiebe, \& Hoffmann, 2005, p.1). Sentiment analysis allows to extract information that usually a human being could not do, due to the amount of text, or even the difficulty to recognizing unstructured patterns.\\	

Another paper written by Nyman, Kapadia, \& Tuckett (2021) considered the possibility of a sentiment index, based in social media, for measure of excitement or anxiety about the financial conjunctural situation. The index would work as a proxy for the market sentiment: bullish or bearish. The ratio of the index would be, then, compared with historical events and others financial indicators (Nyman et al. 2021).\\

Gradually, technics and tools such as sentiment analysis and text mining has been used with implementations in the economic field. It is not hard to find correlations between macroeconomic variables and sentiment analysis extracted from media (Ostapenko, 2020) or even parliament hearings from central banks (Fraccaroli \& Giovannini, 2020).\\

This work has as its main objective the analysis and investigation of how textual patterns provided by the ECB can indicate and correlate with conjunctural and structural moments in the economy (especially in macroeconomics). From structured data taken from press conferences of the European central bank and monetary policy resolutions, understand how “words” and “expressions” can indicate a period of recession, or even verify the credibility of the central bank from its “expressions”. NLP techniques have been increasingly used in order to capture patterns practically imperceptible to the human eye, the use of this technique in the economic field could be of great advancement and usefulness for a better understanding of what central banks are really indicating (through textual documents), and not only reduce the indication of the central bank based on its own resolutions.\\

The greatest relevance and contribution of this work is the use of ECB press conferences as an indicator of sentiments - not as done before, where the main references came from monetary policy reports or speeches. Starting from this point, it will be possible to compare the results obtained with previous works so that an “asymptotic” relationship in terms of results is expected.\\

This final report is fundamentally divided according to the division of the final dissertation, with the exception of the parts referring to the final results and conclusion. Section 2 addresses the literature review, fundamentally alluding to text mining, sentiment analysis, and econometric applications; section 3 deals with the methodology used in the work, reviewing concepts of more computational approaches such as web scraping and NLP techniques; section 4 focuses on the division of the work plan as well as a provisional index of the dissertation; section 5 presents the textual references of this work.

