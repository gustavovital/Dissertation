\chapter{\textbf{Introduction}}  \label{introduction}
\pagenumbering{arabic}

Economic decisions, in particular monetary policies, are based on resolutions provided by central banks. These remedies, however, are not always obvious, especially for lay people. For example, central bank minutes often include much more information than appears at first glance, especially when a group of texts is taken into account for a conjunctive or structural analysis. Understanding a collection of texts can occasionally be very difficult for a human being; depending on the purpose of understanding, the analysis can involve thousands or even millions of pages. However, thanks to developments in computing, questions like these are increasingly likely to be resolved. For a better understanding and knowledge of what a text, or a series of texts (corpus), actually says or expresses, techniques and tools like Natural Language Processing (NLP) and text mining are being heavily used.\\

It is already possible to discover trends in search engines such as Google and Yahoo using text analysis to anticipate macroeconomic indicators or even increase the understanding of consumer behavior. In recent study, \cite{bholat2015text} considers that, although the advances made in the field of computing are significant, the applicability of text mining in the economy is still not used as it could be. In the same study, the authors present cases of applicability of text mining in the economic field, with examples in the job market or even explaining how Natural Language Processing techniques can be applied to economics.\\

An alternative approach of application still in the field of Natural Language Processing is the sentiment analysis. Basically, it is “the task of identifying positive and negative opinions, emotions, and evaluations” \cite[]{wilson2005}. Sentiment analysis allows extracting information that normally a human being would not be able to, due to the amount of text, or even the difficulty of recognizing unstructured patterns. Through sentiment lexicons, it is possible to classify whether a text, phrase or words has a positive, neutral or negative expression (polarity-based lexicon) -- or even assign scores to a text, phrase or word (valence-based lexicon). When working with a set of economic texts, it is possible, then, to assign scores so that feelings can be monitored in a temporal way.\\	

Another article written by \cite{NYMAN2021104119} considered the possibility of a sentiment index, based on social media and networks, for a measure of excitement or anxiety about the financial and economic situation. The index proposes to act as a proxy for market sentiment: bullish or bearish. The ratio of the index would then be compared against historical events and other financial indicators \cite[]{NYMAN2021104119}. The idea of sentiment indices does not come from \cite{NYMAN2021104119} -- authors have already presented the possibility of modeling the economy by including non-observable variables that can represent economic sentiment. Even though in studies such as \cite{barsky2012information, angeletos2013sentiments, akerlof2010animal, gennaioli2018crisis, bholat2015text} the idea of economic sentiments has already been presented; Previous studies \cite{shapiro2020measuring, shapiro2021taking} presented the idea of an index of sentiments coming from Natural Language Processing techniques, where the index presents a boost to economic activity.\\

Gradually, Natural Language Processing techniques and tools have been increasingly used with implementations in the economic field. It is not difficult to find correlations between macroeconomic variables and sentiment analyzes extracted from the media, economic newspapers or magazines \cite[]{ostapenko2020macroeconomic, NYMAN2021104119} or even from speeches or parliamentary hearings of central banks \cite[]{fraccaroli2020central, shapiro2020measuring, shapiro2021taking}.\\

The examination and investigation of how textual patterns supplied by the European Central Bank can point to and connect with conjunctural and structural moments of the economy is the major goal of this work. From the speeches of the European Central Bank, an index of economic sentiment was elaborated that can reflect what the economic scenario goes through -- that is, the reflection of the economy from the point of view of the economic speeches of the European Central Bank. The use of Natural Language Processing techniques in the economics field can be very advanced and useful for a better understanding of what central banks are actually indicating (through textual documents), and not just reduce the central bank's indication based on its own resolutions. Language Processing techniques are being used more and more to capture patterns that are practically imperceptible to the human eye.\\

The greatest relevance and contribution of this work is the use of the speeches of the European Central Bank as an indicator of economic sentiments. From this point, it will be possible to compare the results obtained with previous correlated works in the literature so that an “asymptotic” relationship is expected in terms of results -- in terms of counter intuition, the objective of this work is to corroborate the indicated literature.\\

The chapter \ref{chapter:lit} presents the bibliographic review of this work and exposes the main works used as a reference for this bibliography. The chapter exposes the point of view of \cite{shapiro2020measuring, shapiro2021taking} articles that made use of Natural Language Processing and Sentiment Analysis techniques. Also, other articles \cite{barsky2012information, angeletos2013sentiments} are exposed that work with the \textit{concept} of feelings in an economic scenario, even when considering a general equilibrium scenario.\\

The \ref{cap:methodology} chapter presents the review and introduction of the reference methodology, related to the Natural Language Processing parts and lexicons. Two fundamental lexicons are exposed and used in this work -- VADER \cite[]{hutto2014vader} and LM-SA-2020 \cite[]{lmdata}, the first lexicon being a valence-based lexicon and the second a polarity- based lexicon, economically grounded. Still, concepts of Natural Language Processing and Sentiment Analysis are defined.\\

The \ref{cap:results} chapter presents the result of the present work. This chapter addresses the case study carried out. Using the speeches of the European Central Bank as a basis, two studies are carried out: in the first, models of selection of variables are estimated to infer whether indices of economic sentiments based on NLP would be statistically significant for the estimation of economic variables; and in the second study we estimate the responses of the impulse response functions given a shock in sentiment indexes using an Autoregressive Vector -- this would make it possible to understand how the estimated economic variables would behave when a ``variation'' of sentiments occurs in the economic scenario.\\

The work ends with the fifth chapter, highlighting the final considerations of the dissertation and the conclusions regarding what was done.
